\section{Diffusion Kernels}
\subsection{Motivation}
In this section it will be elaborated on why global structures need not be preserved and how this leads to diffusion processes.
\subsubsection{Preservation of Global Structures}
Suppose data points $x_i \in \Real^k$ are generated by a low dimensional parameter $\theta_i \in \Real^{k'}$, $k' \ll k$ given a map $\Phi: \Real^{k'} \rightarrow \Real^k$. One problem is the (numerical) smoothness of $\Phi$ which is not necessarily given.
Another problem is that for large $k$ the euclidian distance is no longer a meaningful measure as the volume of the $k$-dimensional unit ball $\frac{\pi^\frac{k}{2}}{\Gamma(\frac{k}{2}+1)}$ converges to $0$ as $lim_{k\rightarrow\infty}$ which is also known as the \textit{curse of dimensionality}. One may conclude that large distances in the ambient space need not necessarily be preserved as they do not hold much information except that $x_i$ is not ``very close'' to $x_j$.

Analogous to Riemannian manifolds (where metric tensors, inducing an inner product on the tangent space and a metric via the exponential map, define the manifold's geometry) we focus solely on local distances in order to recover intrinsic global structures.

\subsubsection{A Dual Approach}
From inverse problems in spectral geometry (e.g. ``Can One Hear the Shape of a Drum?'') it is known that much\footnote{the most famous example being Weyl's proof of $\#\{\lambda_k : \lambda_k < \lambda\} \approx \frac{\text{area}(\Gamma)}{2\pi}\lambda$ as $\lambda \rightarrow \infty$. (\cite{Canzani2013})} of the geometry of a given set $\Gamma$ can be derived from the analysis of functions defined on $\Gamma$.

In this work eigenvalues and eigenfunctions of averaging operators, i.e., operators whose kernel corresponds to transition probabilities of a Markov process, will be studied in order to define a diffusion map which embeds the data into a Euclidian space where the Euclidian distance is just the diffusion metric.

\subsection{Construction of a Random Walk on the Data}
\subsubsection{Definitions}
Let $(\Gamma, \mathcal{A}, \mu)$ be a measure space. In practical applications $\Gamma$ is the given data set consisting of finitely many data points and $\mu$ is the counting measure to represent the distribution of the points in the data set. In addition, suppose we are given a symmetric kernel $k: \Gamma \times \Gamma \rightarrow \Real^+$ which defines the local geometry of $\Gamma$.

\subsubsection{Examples}
Usually $\Gamma$ is either a subset of the Euclidian space or a weighted graph.

In the first case it seems natural to write $k$ as a function of the Euclidian distance $\nu(||x-y||)$.

In the second case let $b(x,y)$ be the associated adjacancy matrix, that is, $b(x,y) = 1$ if there is an edge going from $x$ to $y$, and $b(x,y) = 0$ otherwise. The kernel $b$ defines a notion of neighborhood for each point, and also a non-symmetric distance given by $1-b(x,y)$. Clearly $b$ is not symmetric in general, but we can consider
\begin{equation*}
k_1(x,y):=\int_\Gamma b(x,u)b(y,u)d\mu(u)
\end{equation*}
\begin{equation*}
k_2(x,y):=\int_\Gamma b(u,x)b(u,y)d\mu(u)
\end{equation*}
where $k_1(x,y)$ counts the number of common neighbors to $x$ and $y$, while $k_2(x,y)$ counts the number of nodes for which $x$ and $y$ are common neighbors.

\subsubsection{Normalized Graph Laplacian Construction}
Generally, such a kernel represents some notion of affinity between points of $\Gamma$ and thus one can think of the data points as being the nodes of a symmetric graph whose weight function is specified by $k$. From the graph defined by $(\Gamma, k)$, one can construct a reversible Markov chain on $\Gamma$.

To normalize the kernel, define
\begin{definition}\label{def:normFunc}
$$v^2(x)=\int_\Gamma k(x,y)d\mu(y)$$
\end{definition}
and
\begin{definition}
$$p(x,y)=\frac{k(x,y)}{v^2(x)}.$$
\end{definition}
$p(x,y)$ is no longer symmetric, but inherited the positivity and now satisfies a conservation property:

\begin{equation*}
\int_\Gamma p(x,y)d\mu(y)=1
\end{equation*}

As a result the matrix $P:=(p(i,j))_{i,j}$ is stochastic and can be interpreted as the transition matrix of a homogeneous Markov process on $\Gamma$. In spectral graph theory $\mathbb{I}-P$ is commonly referred to as normalized, weighted graph Laplacian. This naming is justified in \ref{thm:diffOpExpansion}\todo{better double check that claim in the footnote}\footnote{Let w.l.o.g. $m_0=1$ and $m_2=2$. Now set $L_\eps = \frac{\eps I - G_\eps}{\eps}$; for uniformly sampled data the potential term $\omega$ vanishes and $\lim_{\eps\rightarrow 0} L_\eps = \Delta$}.

To investigate the spectral properties of the corresponding integral operator $P$ defined by $Pf(x)=\int_\Gamma p(x,y)f(y)d\mu(y)$ it is beneficial to examine the symmetric, conjugated Operator $A$.

\begin{definition}
Let $$a(x,y)=\frac{k(x,y)}{v(x)v(y)}=v(x)p(x,y)\frac{1}{v(y)},$$ then the corresponding diffusion operator $A$ is defined as $$Af(x)=\int_\Gamma a(x,y)f(y)d\mu(y)$$
\end{definition}

Notice that by definition of $a$ one obtains a symmetric form and thus a symmetric operator $A$.

\subsection{Diffusion Kernels}
\begin{theorem}[Spectral Properties of the Diffusion Operator]
The diffusion operator $A$ with kernel $a$ is bounded from $L^2(\Gamma, d\mu)$ into itself, symmetric and positive semi-definite.

Moreover, its norm is $$||A||=1$$ and is taken by the eigenfunction $$Av=v.$$
\end{theorem}
\begin{proof}
Let $f\in L^2(\Gamma, d\mu)$. We have:
\begin{equation}\label{eq:positivityOfA}
\langle Af,f \rangle = \int_{\Gamma^2}k(x,y)\frac{f(x)}{v(x)}\frac{f(y)}{v(y)}d\mu(x)d\mu(y).
\end{equation}

Applying the Cauchy-Schwartz inequality we get:
\begin{equation*}\begin{array}{l l}
\left|\bigintsss_\Gamma k(x,y)\frac{f(y)}{v(y)}d\mu(y)\right| &= 
\left(\bigintsss_\Gamma k(x,y)d\mu(y)\right)^{\frac{1}{2}}
\left(\bigintsss_\Gamma k(x,y)\frac{f(y)^2}{v(y)^2} d\mu(y)\right)^{\frac{1}{2}} \\
&=v(x)(\bigintsss_\Gamma k(x,y)\frac{f(y)^2}{v(y)^2} d\mu(y))^{\frac{1}{2}}
\end{array}\end{equation*}

Hence:
$$\langle Af,f \rangle \leq \int_\Gamma |f(x)|\left(\int_\Gamma k(x,y)\frac{f(y)^2}{v(y)^2} d\mu(y)\right)^{\frac{1}{2}}d\mu(x)$$
and by using the Cauchy-Schwartz inequality once again:
$$\langle Af,f \rangle \leq ||f||\left(\int_{\Gamma^2} \frac{k(x,y)}{v(y)^2}f(y)^2 d\mu(y)d\mu(x)\right)^{\frac{1}{2}} = ||f||^2$$ by symmetry of the kernel which, in combination with \eqref{eq:positivityOfA}, also implies the positivity of $A$.

Plugging in $v$ for $f$ it follows immediately that the eigenvalue $1$ is actually obtained and $v$ is an eigenfunction.
\end{proof}

\begin{theorem}[Spectral Decomposition of the Diffusion Kernel]
Assuming $A$ is compact\footnote{which is no constraint in practice since data is finite}\todo{try to generalize lemma 3.4 of \cite{Teschl2014} (guess it won't work though)} and $A\phi_l=\lambda_l\phi_l$ we may write the kernel as
\begin{equation*}
a(x,y)=\sum_{l\geq 0}\lambda_l\phi_l(x)\phi_l(y)
\end{equation*}
with $\lambda_0 = 1$ and $\lim_{l\rightarrow\infty}\lambda_l = 0$ monotonically.
\end{theorem}
\begin{proof}
First, note that $A$ being compact implies that the spectrum is discrete and the sum thus is well defined. By $A$ being symmetric and compact the spectral theorem applies and we get that there exists a sequence of real eigenvalues $\lambda_l$ converging to $0$. The corresponding normalized eigenvectors $\phi_l$ form an orthonormal set and every $f\in L^2(\Gamma,d\mu)$ can be written as
$$f=\sum_{l\geq 0}\langle \phi_l,f\rangle \phi_l + h$$
where $h\in Ker(A)$.

It follows that
$$Af(x)=\int_\Gamma a(x,y)f(y)d\mu(y) = \sum_{l\geq 0}\lambda_l \int_\Gamma \phi_l(y)f(y)d\mu(y)\ \phi_l(x)$$
which, by linearity of the integral and comparison of components\todo{``Komponentenvergleich'' auf Englisch finden}, is just what we were looking for.
\end{proof}
From definition of $a(x,y)$ we see that 
\begin{equation}\label{eq:spectralDecompositionOfP}
p(x,y)=\sum_{l\geq 0}\lambda_l\underbrace{\frac{\phi_l(x)}{v(x)}}_{=:\psi_l(x)}\phi_l(y)v(y)
\end{equation}
 which enables us to efficiently compute $t$th powers $p_t$ of $p$.

There are two ways to interpret $p_t$:
\begin{enumerate}
\item $p_t$ has a probabilistic interpretation as the probability for a Markov chain with transition matrix $P$ to reach $y$ from $x$ in $t$ steps.
\item the dual point of view is that of the functions defined on the data. The kernel $p_t$ can be viewed as a bump or more precisely, if $x\in\Gamma$ is fixed, then $p_t(x, \cdotp)$ is a bump function centered at $x$ and of width increasing with $t$ which intuitively captures the idea of diffusion.
\end{enumerate}

\subsection{Embedding in the Euclidian Space}
\begin{definition}
Let $$D_t(x,y)^2 =
||p_t(x, \cdotp) - p_t(y, \cdotp)||_{L^2(\Gamma, d\mu/v)}^2 =
\int_\Gamma \big(p_t(x, u) - p_t(y, u)\big)^2 \frac{d\mu(u)}{v(u)}$$
be the family of \textit{diffusion distances} parameterized by $t$.
\end{definition}

For a fixed value of $t$ $D_t$ defines a distance on the set $\Gamma$ which is small only if there is a large number of small paths connecting $x$ and $y$ (i.e. if there is a large probability of getting from $x$ to $y$ in $t$ steps). It thus emphasizes the notion of a cluster.

Another property following from the summation over all possible paths is that this distance is very robust to noise perturbation (in contrast to the geodesic distance).

\begin{theorem}[A Numerically Feasible Representation]
$$D_t(x,y)=\left(\sum_{l\geq 0} \left(\lambda_l^t(\psi_l(x) - \psi_l(y))\right)^2 \right)^{\frac{1}{2}}$$
\end{theorem}

\begin{proof}
$\{\phi_l\}_{l\geq 0}$ forming an orthonormal basis for $L^2(\Gamma, d\mu)$ implies that $\{\phi_l v\}_{l\geq 0}$ is an orthonormal basis for $L^2(\Gamma, d\mu/v)$ and thus for $x$ fixed \eqref{eq:spectralDecompositionOfP} may be seen as orthogonal expansion of the function $y \mapsto p_t(x,y)$ into the basis $\{\phi_l v\}_{l\geq 0}$.
The coefficients are given by $\{\lambda_l^t \psi_l(x)\}_{l\geq 0}$. The statement follows directly using the Pythagorean theorem.
\end{proof}

An imidiate consequence is that the diffusion distance is well approximable and that it converges towards a function of (numerical) rank $1$ as $t\rightarrow\infty$ because of the vanishing influence of all eigenvectors with eigenvalues $<1$.

One possible interpretation is that $D_t(x,y)$ measures the distance between bumps of ``magnitude'' $t$ being centered around two points $x$ and $y$. As $t$ gets larger so does the size of the supports and the number of eigenfunctions needed to calculate $D_t(x,y)$ decreases. This number is related to the minimum number of bumps necessary to cover the set X (like in Weyl’s asymptotic law for the decay of the spectrum).

In order to calculate $D_t(x,y)$ to a preset accuracy $\delta>0$ with a finite number of terms we set
$$s_t(\delta)= \text{max}\{l\in\mathbf{N} : \lambda_l^t \>> \delta\lambda_1^t\}$$
so that, up to relative precision $\delta$
\begin{equation}\label{eq:DtApproximation}
D_t(x,y)=\left(\sum_{l=0}^{s_t(\delta)} \left(\lambda_l^t(\psi_l(x) - \psi_l(y))\right)^2 \right)^{\frac{1}{2}}.
\end{equation}

\begin{definition}
Let $\{\Psi_t\}_{t\in\mathbf{N}}$,
$$\Psi_t(x)=\begin{pmatrix}
  \lambda_1^{t}\psi_1(x) \\
  \lambda_2^{t}\psi_2(x) \\
  \vdots \\
  \lambda_{s_t(\delta)}^{t}\psi_{s_t(\delta)}(x)
 \end{pmatrix}$$ be the family of diffusion maps. Each component of $\Psi_t(x)$ is termed diffusion coordinate.
\end{definition}

According to \eqref{eq:DtApproximation} diffusion maps embed data in a Euclidian space in such a way that the Euclidian distance equals the diffusion distance up to a relative error $\delta$.

