\section{Definitions}
finite graph (henceforth with n vertices)

simple graph: an undirected graph that has no loops (edges connected at both ends to the same vertex) and no more than one edge between any two different vertices

degree matrix: a diagonal matrix which contains information about the degree of each vertex—that is, the number of edges attached to each vertex; undirected graph $\rightarrow$ symmetric

adjacancy matrix: $n \times n$ matrix where the non-diagonal entry $a_{ij}$ is the number of edges from vertex $i$ to vertex $j$, and the diagonal entry $a_{ii}$, depending on the convention, is either once or twice the number of edges (loops) from vertex $i$ to itself

Laplacian matrix $L:=D-A$ where $D$ is the graph's degree matrix and $A$ the adjacancy matrix

\section{Properties}
\begin{itemize}
    \item positive-semidefinite matrix
    \item eigenvalues
    \begin{itemize}
        \item 0 always eigenvalue
        \item importance of second smallest eigenvalue
        \item corresponding eigenfunctions may be seen as a generalization of (low frequency) Fourier-transformation
    \end{itemize}
    \item operator on the $n$-dimensional vector space of functions $f:V \rightarrow R$, where $V$ is the vertex set of $G$, and $n=|V|$
\end{itemize}


\section{Interpretation as the Discrete Laplace Operator}