\documentclass[11pt]{report}
\usepackage[utf8]{inputenc}
\usepackage{amssymb,amsthm,mathtools}
\usepackage{hyperref,graphicx}
\usepackage{bigints}
\usepackage{todonotes}
\usepackage{natbib}
\graphicspath{ {images/} }

\newcommand{\Man}{\mathcal{M}}
\newcommand{\Real}{\mathbb{R}}
\newcommand{\eps}{\varepsilon}

\newtheorem{definition}{Definition}[chapter]
\newtheorem{theorem}{Theorem}[section]
\newtheorem{corollary}{Corollary}[theorem]
\newtheorem{lemma}[theorem]{Lemma}
\newtheorem*{remark}{Remark}

\begin{document}

\title{
	{Diffusion Maps}\\
	{\large University of Vienna}\\
	{\includegraphics{university.png}}
}
\author{Peter Mühlbacher}
\date{October 1, 2014}
\maketitle

\begin{abstract}
In \autoref{chap:diffusionMaps} a new distance, which involves summing over all possible paths between two points and is thus very robust to noise perturbation, will be introduced. In order to efficiently compute this distance up to a given relative error $\delta$, its kernel will be decomposed over an orthonormal basis of eigenfunctions of the corresponding integral operator. It will then be shown that this procedure can be used to find a meaningful embedding of data in the Euclidian space.

In \autoref{chap:fokkerPlanck} density and time will also be taken into account which leads to diffusion processes and their connection to stochastic processes. This can also be used to study the long-time behaviour of important stochastic systems by investigating their lower dimensional representations.
\end{abstract}

\tableofcontents

\chapter*{Introduction}
The contentual layout follows that of \cite{Coifman20065} and \cite{Belkin2003}. Some definitions, theorems and examples will be taken from these works as well.

\section{Dimensionality Reduction}

\chapter{Diffusion Maps}\label{chap:diffusionMaps}
\section{Diffusion Kernels}
\subsection{Motivation}
In this section it will be elaborated on why global structures need not be preserved and how this leads to diffusion processes.
\subsubsection{Preservation of Global Structures}
Suppose data points $x_i \in \Real^k$ are generated by a low dimensional parameter $\theta_i \in \Real^{k'}$, $k' \ll k$ given a map $\Phi: \Real^{k'} \rightarrow \Real^k$. One problem is the (numerical) smoothness of $\Phi$ which is not necessarily given.
Another problem is that for large $k$ the euclidian distance is no longer a meaningful measure as the volume of the $k$-dimensional unit ball $\frac{\pi^\frac{k}{2}}{\Gamma(\frac{k}{2}+1)}$ converges to $0$ as $lim_{k\rightarrow\infty}$ which is also known as the \textit{curse of dimensionality}. One may conclude that large distances in the ambient space need not necessarily be preserved as they do not hold much information except that $x_i$ is not ``very close'' to $x_j$.

Analogous to Riemannian manifolds (where metric tensors, inducing an inner product on the tangent space and a metric via the exponential map, define the manifold's geometry) we focus solely on local distances in order to recover intrinsic global structures.

\subsubsection{A Dual Approach}
From inverse problems in spectral geometry (e.g. ``Can One Hear the Shape of a Drum?'') it is known that much\footnote{the most famous example being Weyl's proof of $\#\{\lambda_k : \lambda_k < \lambda\} \approx \frac{\text{area}(\Gamma)}{2\pi}\lambda$ as $\lambda \rightarrow \infty$. (\cite{AnalysisOnManifolds})} of the geometry of a given set $\Gamma$ can be derived from the analysis of functions defined on $\Gamma$.

In this work eigenvalues and eigenfunctions of averaging operators, i.e., operators whose kernel corresponds to transition probabilities of a Markov process, will be studied in order to define a diffusion map which embeds the data into a Euclidian space where the Euclidian distance is just the diffusion metric.

\subsection{Construction of a Random Walk on the Data}
\subsubsection{Definitions}
Let $(\Gamma, \mathcal{A}, \mu)$ be a measure space. In practical applications $\Gamma$ is the given data set consisting of finitely many data points and $\mu$ is the counting measure to represent the distribution of the points in the data set. In addition, suppose we are given a symmetric kernel $k: \Gamma \times \Gamma \rightarrow \Real^+$ which defines the local geometry of $\Gamma$.

\subsubsection{Examples}
Usually $\Gamma$ is either a subset of the Euclidian space or a weighted graph.

In the first case it seems natural to write $k$ as a function of the Euclidian distance $\nu(||x-y||)$.

In the second case let $b(x,y)$ be the associated adjacancy matrix, that is, $b(x,y) = 1$ if there is an edge going from $x$ to $y$, and $b(x,y) = 0$ otherwise. The kernel $b$ defines a notion of neighborhood for each point, and also a non-symmetric distance given by $1-b(x,y)$. Clearly $b$ is not symmetric in general, but we can consider
\begin{equation*}
k_1(x,y):=\int_\Gamma b(x,u)b(y,u)d\mu(u)
\end{equation*}
\begin{equation*}
k_2(x,y):=\int_\Gamma b(u,x)b(u,y)d\mu(u)
\end{equation*}
where $k_1(x,y)$ counts the number of common neighbors to $x$ and $y$, while $k_2(x,y)$ counts the number of nodes for which $x$ and $y$ are common neighbors.

\subsubsection{Normalized Graph Laplacian Construction}
Generally, such a kernel represents some notion of affinity between points of $\Gamma$ and thus one can think of the data points as being the nodes of a symmetric graph whose weight function is specified by $k$. From the graph defined by $(\Gamma, k)$, one can construct a reversible Markov chain on $\Gamma$.

To normalize the kernel, define
\begin{definition}\label{def:normFunc}
$$v^2(x)=\int_\Gamma k(x,y)d\mu(y)$$
\end{definition}
and
\begin{definition}
$$p(x,y)=\frac{k(x,y)}{v^2(x)}.$$
\end{definition}
$p(x,y)$ is no longer symmetric, but inherited the positivity and now satisfies a conservation property:

\begin{equation*}
\int_\Gamma p(x,y)d\mu(y)=1
\end{equation*}

As a result the matrix $P:=(p(i,j))_{i,j}$ is stochastic and can be interpreted as the transition matrix of a homogeneous Markov process on $\Gamma$. In spectral graph theory $\mathbb{I}-P$ is commonly referred to as normalized, weighted graph Laplacian. This naming is justified in \ref{diffOpExpansion}\todo{better double check that claim in the footnote}\footnote{Let w.l.o.g. $m_0=1$ and $m_2=2$. Now set $L_\eps = \frac{\eps I - G_\eps}{\eps}$; for uniformly sampled data the potential term $\omega$ vanishes and $\lim_{\eps\rightarrow 0} L_\eps = \Delta$}.

To investigate the spectral properties of the corresponding integral operator $P$ defined by $Pf(x)=\int_\Gamma p(x,y)f(y)d\mu(y)$ it is beneficial to examine the symmetric, conjugated Operator $A$.

\begin{definition}
Let $$a(x,y)=\frac{k(x,y)}{v(x)v(y)}=v(x)p(x,y)\frac{1}{v(y)},$$ then the corresponding diffusion operator $A$ is defined as $$Af(x)=\int_\Gamma a(x,y)f(y)d\mu(y)$$
\end{definition}

Notice that by definition of $a$ one obtains a symmetric form and thus a symmetric operator $A$.

\subsection{Diffusion Kernels}
\begin{theorem}[Spectral Properties of the Diffusion Operator]
The diffusion operator $A$ with kernel $a$ is bounded from $L^2(\Gamma, d\mu)$ into itself, symmetric and positive semi-definite.

Moreover, its norm is $$||A||=1$$ and is taken by the eigenfunction $$Av=v.$$
\end{theorem}
\begin{proof}
Let $f\in L^2(\Gamma, d\mu)$. We have:
\begin{equation}\label{positivityOfA}
\langle Af,f \rangle = \int_{\Gamma^2}k(x,y)\frac{f(x)}{v(x)}\frac{f(y)}{v(y)}d\mu(x)d\mu(y).
\end{equation}

Applying the Cauchy-Schwartz inequality we get:
\begin{equation*}\begin{array}{l l}
\left|\int_\Gamma k(x,y)\frac{f(y)}{v(y)}d\mu(y)\right| &= 
\left(\int_\Gamma k(x,y)d\mu(y)\right)^{\frac{1}{2}}
\left(\int_\Gamma k(x,y)\frac{f(y)^2}{v(y)^2} d\mu(y)\right)^{\frac{1}{2}} \\
&=v(x)(\int_\Gamma k(x,y)\frac{f(y)^2}{v(y)^2} d\mu(y))^{\frac{1}{2}}
\end{array}\end{equation*}

Hence:
$$\langle Af,f \rangle \leq \int_\Gamma |f(x)|\left(\int_\Gamma k(x,y)\frac{f(y)^2}{v(y)^2} d\mu(y)\right)^{\frac{1}{2}}d\mu(x)$$
and by using the Cauchy-Schwartz inequality once again:
$$\langle Af,f \rangle \leq ||f||\left(\int_{\Gamma^2} \frac{k(x,y)}{v(y)^2}f(y)^2 d\mu(y)d\mu(x)\right)^{\frac{1}{2}} = ||f||^2$$ by symmetry of the kernel which, in combination with \eqref{positivityOfA}, also implies the positivity of $A$.

Plugging in $v$ for $f$ it follows immediately that the eigenvalue $1$ is actually obtained and $v$ is an eigenfunction.
\end{proof}

\begin{theorem}[Spectral Decomposition of the Diffusion Kernel]
Assuming $A$ is compact\footnote{which is no constraint in practice since data is finite}\todo{try to generalize lemma 3.4 of \cite{FA_SCRIPT} (guess it won't work though)} and $A\phi_l=\lambda_l\phi_l$ we may write the kernel as
\begin{equation*}
a(x,y)=\sum_{l\geq 0}\lambda_l\phi_l(x)\phi_l(y)
\end{equation*}
with $\lambda_0 = 1$ and $\lim_{l\rightarrow\infty}\lambda_l = 0$ monotonically.
\end{theorem}
\begin{proof}
First, note that $A$ being compact implies that the spectrum is discrete and the sum thus is well defined. By $A$ being symmetric and compact the spectral theorem applies and we get that there exists a sequence of real eigenvalues $\lambda_l$ converging to $0$. The corresponding normalized eigenvectors $\phi_l$ form an orthonormal set and every $f\in L^2(\Gamma,d\mu)$ can be written as
$$f=\sum_{l\geq 0}\langle \phi_l,f\rangle \phi_l + h$$
where $h\in Ker(A)$.

It follows that
$$Af(x)=\int_\Gamma a(x,y)f(y)d\mu(y) = \sum_{l\geq 0}\lambda_l \int_\Gamma \phi_l(y)f(y)d\mu(y)\ \phi_l(x)$$
which, by linearity of the integral and comparison of components\todo{``Komponentenvergleich'' auf Englisch finden}, is just what we were looking for.
\end{proof}
From definition of $a(x,y)$ we see that 
\begin{equation}\label{spectralDecompositionOfP}
p(x,y)=\sum_{l\geq 0}\lambda_l\underbrace{\frac{\phi_l(x)}{v(x)}}_{=:\psi_l(x)}\phi_l(y)v(y)
\end{equation}
 which enables us to efficiently compute $t$th powers $p_t$ of $p$.

There are two ways to interpret $p_t$:
\begin{enumerate}
\item $p_t$ has a probabilistic interpretation as the probability for a Markov chain with transition matrix $P$ to reach $y$ from $x$ in $t$ steps.
\item the dual point of view is that of the functions defined on the data. The kernel $p_t$ can be viewed as a bump or more precisely, if $x\in\Gamma$ is fixed, then $p_t(x, \cdotp)$ is a bump function centered at $x$ and of width increasing with $t$ which intuitively captures the idea of diffusion.
\end{enumerate}

\subsection{Embedding in the Euclidian Space}
\begin{definition}
Let $$D_t(x,y)^2 =
||p_t(x, \cdotp) - p_t(y, \cdotp)||_{L^2(\Gamma, d\mu/v)}^2 =
\int_\Gamma \big(p_t(x, u) - p_t(y, u)\big)^2 \frac{d\mu(u)}{v(u)}$$
be the family of \textit{diffusion distances} parameterized by $t$.
\end{definition}

For a fixed value of $t$ $D_t$ defines a distance on the set $\Gamma$ which is small only if there is a large number of small paths connecting $x$ and $y$ (i.e. if there is a large probability of getting from $x$ to $y$ in $t$ steps). It thus emphasizes the notion of a cluster.

Another property following from the summation over all possible paths is that this distance is very robust to noise perturbation (in contrast to the geodesic distance).

\begin{theorem}[A Numerically Feasible Representation]
$$D_t(x,y)=\left(\sum_{l\geq 0} \left(\lambda_l^{l}(\psi_l(x) - \psi_l(y))\right)^2 \right)^{\frac{1}{2}}$$
\end{theorem}

\begin{proof}
$\{\phi_l\}_{l\geq 0}$ forming an orthonormal basis for $L^2(\Gamma, d\mu)$ implies that $\{\phi_l v\}_{l\geq 0}$ is an orthonormal basis for $L^2(\Gamma, d\mu/v)$ and thus for $x$ fixed \eqref{spectralDecompositionOfP} may be seen as orthogonal expansion of the function $y \mapsto p_t(x,y)$ into the basis $\{\phi_l v\}_{l\geq 0}$.
The coefficients are given by $\{\lambda_l^t \psi_l(x)\}_{l\geq 0}$. The statement follows directly using the Pythagorean theorem.
\end{proof}

An imidiate consequence is that the diffusion distance is well approximable and that it converges towards a function of (numerical) rank $1$ as $t\rightarrow\infty$ because of the vanishing influence of all eigenvectors with eigenvalues $<1$.

One possible interpretation is that $D_t(x,y)$ measures the distance between bumps of ``magnitude'' $t$ being centered around two points $x$ and $y$. As $t$ gets larger so does the size of the supports and the number of eigenfunctions needed to calculate $D_t(x,y)$ decreases. This number is related to the minimum number of bumps necessary to cover the set X (like in Weyl’s asymptotic law for the decay of the spectrum).

In order to calculate $D_t(x,y)$ to a preset accuracy $\delta>0$ with a finite number of terms we set
$$s_t(\delta)= \text{max}\{l\in\mathbf{N} : \lambda_l^t \>> \delta\lambda_1^t\}$$
so that, up to relative precision $\delta$
\begin{equation}\label{DtApproximation}
D_t(x,y)=\left(\sum_{l=0}^{s_t(\delta)} \left(\lambda_l^{l}(\psi_l(x) - \psi_l(y))\right)^2 \right)^{\frac{1}{2}}.
\end{equation}

\begin{definition}
Let $\{\Psi_t\}_{t\in\mathbf{N}}$,
$$\Psi_t(x)=\begin{pmatrix}
  \lambda_1^{t}\psi_1(x) \\
  \lambda_2^{t}\psi_2(x) \\
  \vdots \\
  \lambda_{s_t(\delta)}^{t}\psi_{s_t(\delta)}(x)
 \end{pmatrix}$$ be the family of diffusion maps. Each component of $\Psi_t(x)$ is termed diffusion coordinate.
\end{definition}

According to \eqref{DtApproximation} diffusion maps embed data in a Euclidian space in such a way that the Euclidian distance equals the diffusion distance up to a relative error $\delta$.



\chapter{Connection with the Fokker-Planck Equation}\label{chap:fokkerPlanck}
\section{A Family of Anisotropic Diffusion Maps}\todo{define FPE, infinitesimal generators, SDEs, etc.}

In this section we investigate the probability space $(X, \mathcal{A}, \mu)$ and therefore assume an infinite number of data points whose distribution is determined by the probability measure $\mu(x)=e^{-U(x)}$. $U$ may be interpreted as a potential function.\todo{elaborate on probability vs. potential}

\subsection{Construction of the Family of Diffusions}
Start with a Gaussian kernel $k_\eps(x,y)=e^{-\frac{||x-y||^2}{\eps}}$ and let $\alpha > 0$ being a parameter indexing this family of diffusions.

We can estimate the local density $q_\eps$ by 

\begin{definition}
$$q_\eps(x)=\int_X k_\eps(x,y)d\mu(y)$$
\end{definition}

Now consider the family of kernels

\begin{definition}
$$k_\eps^{(\alpha)}(x,y)=\frac{k_\eps(x,y)}{q_\eps^\alpha(x)q_\eps^\alpha(y)} $$
\end{definition}

We compute the normalization factor\footnote{Note that this factor is the continuous analogue to definition \ref{def:normFunc}.} $d_\eps^{(\alpha)}$

$$d_\eps^{(\alpha)}(x) = \int_X k_\eps^{(\alpha)}(x,y)d\mu(y)$$

and define the forward and symmetric ($p^{(\alpha)}$ and $a^{(\alpha)}$ respectively) transition probability kernels:

\begin{definition}\begin{equation*}\begin{aligned}
&p_\eps^{(\alpha)}(x|y) = \frac{k_\eps^{(\alpha)}(x,y)}{d_\eps^{(\alpha)}(y)} = \text{Pr}\big(x(t+\eps)=x \, |\, x(t)=y\big)\\
&a_\eps^{(\alpha)}(x|y) = \frac{k_\eps^{(\alpha)}(x,y)}{\sqrt{d_\eps^{(\alpha)}(x)d_\eps^{(\alpha)}(y)}}
\end{aligned}\end{equation*}\end{definition}

Now we define the  forward, backward and symmetric Chapman-Kolmogorov operators on functions defined on this probability space, as follows:

\begin{definition}\begin{equation*}\begin{aligned}
T_{f,\eps}^{(\alpha)}[\phi](x) &= \int_X p_\eps^{(\alpha)}(x|y)\phi(y)d\mu(y)\\
T_{b,\eps}^{(\alpha)}[\phi](x) &= \int_X p_\eps^{(\alpha)}(y|x)\phi(y)d\mu(y)\\
T_{s,\eps}^{(\alpha)}[\phi](x) &= \int_X a_\eps^{(\alpha)}(x,y)\phi(y)d\mu(y)
\end{aligned}\end{equation*}\end{definition}

If $\phi(x)$ is the probability of finding the system at location $x$ at time $t = 0$, then $T_{f,\eps}^{(\alpha)}[\phi]$ is the evolution of this probability to time $t = \eps$. Similarly, if $\psi(z)$ is some function on the space, then $T_{b,\eps}^{(\alpha)}[\psi](x)$ is the mean value of that function at time $\eps$ for a random walk that started at $x$, and so $\left(T_{b,\eps}^{(\alpha)}\right)^m[\psi](x)$ is the average value of the function at time $t = m\eps$.

By definition, the operators $T_{f,\eps}^{(\alpha)}$ and $T_{b,\eps}^{(\alpha)}$ are adjoint under the inner product with weight $\mu$, while the operator $T_{s,\eps}^{(\alpha)}$ is self adjoint under this inner product.

Moreover, just like in the discrete case, since $T_{s,\eps}^{(\alpha)}$ is obtained via conjugation of the kernel $p^{(\alpha)}$ all three operators share the same eigenvalues. The corresponding eigenfunctions can be found via
conjugation by $\sqrt{d^{(\alpha)}_\eps}$; i.e.:
If $T_{s,\eps}^{(\alpha)}\phi_s = \lambda\phi_s$, then the corresponding eigenfunctions for $T_{f,\eps}^{(\alpha)}$ and $T_{b,\eps}^{(\alpha)}$ are $\phi_f =
\sqrt{q_\eps}\phi_s$ and $\phi_b = \frac{\phi_s}{\sqrt{q_\eps}}$, respectively.\todo{double check that section}

Since $\sqrt{q_\eps}$ is the first eigenfunction with $\lambda_0 = 1$ of $T_s^{(\alpha)}$, the steady state of the forward operator is simply $q_\eps(x)$, while for the backward operator it is the uniform density, $\phi_b \equiv 1$.

Furthermore, note that the eigenvalues and eigenvectors of the discrete Markov chain described in the previous section are discrete approximations to the eigenvalues and eigenfunctions of these continuous operators.

\subsection{Transition to Diffusion Processes}
Interpreting $\eps$ as a timestep\footnote{While in the case of a finite amount of data, $\eps$ must remain finite so as to have enough neighbors in a ball of radius $\mathcal{O}(\eps^\frac{1}{2})$ near each point $x$, as the number of samples goes to infinity, we can take smaller and smaller values of $\eps$.} and recalling the asymptotic expansions derived in the appendix it is instructive to look at the limit $\eps \rightarrow 0$. In this case, the transition probability densities of the Markov chain (that is continuous in space, but discrete in time) converge to those of a diffusion process, whose time evolution is described by a differential equation

\begin{equation*}
\frac{\partial \phi}{\partial t} = \mathcal{H}_f^{(\alpha)}\phi
\end{equation*}

where $\mathcal{H}_f^{(\alpha)}$ is the infinitesimal generator or propagator of the forward operator, defined as

\begin{definition}\begin{equation*}
\mathcal{H}_f^{(\alpha)}=\lim_{\eps\rightarrow 0}\frac{I-T_{f,\eps}^{(\alpha)}}{\eps}
\end{equation*}\end{definition}

Similarly, the inifinitesimal operator of the backward operator is given by

\begin{definition}\begin{equation*}
\mathcal{H}_b^{(\alpha)}=\lim_{\eps\rightarrow 0}\frac{T_{b,\eps}^{(\alpha)}-I}{\eps}
\end{equation*}\end{definition}

and as shown in the appendix (theorem \ref{thm:bFPO})

\begin{equation*}
\mathcal{H}_b^{(\alpha)}\psi = \Delta\psi - 2(1-\alpha)\nabla\psi\cdot\nabla U
\end{equation*}
\todo{add case for the fFPO and cases for $\alpha\in\{0,\frac{1}{2},1 \}$}

\appendix
\chapter{Asymptotics for Laplacian Operators}
\todo{write introduction + motivation (latter being the justification for naming it graph ``Laplacian'' earlier)}
In the following we will deal with a compact manifold $\Man$ that is $C^\infty$. Let $x$ be a fixed point, not on the on the boundary, $T_x\Man$ be the tangent space to $\Man$ at $x$ and $(e_1,\dots,e_d)$ be a fixed orthonormal basis of $T_x\Man$. Furthermore two systems of local coordinates will be introduced:
\begin{enumerate}
\item \textit{(Normal coordinates)} The exponential map $exp_x$ generates a set of orthogonal geodesics $(\gamma_1,\dots,\gamma_d)$ intersecting at x with initial velocity $(e_1,\dots,e_d)$. Now every point $y\in\Man$ in a sufficiently small neighborhood of $x$ has a set of \textit{normal coordinates} $(s_1,\dots,s_d)$ along these geodesics.
\item \textit{(Tangent coordinates)} Considering the orthogonal projection $u$ of $y$ on $T_x\Man$, where $u_i = \langle y-x,e_i\rangle$ in $(e_1,\dots,e_d)$, we get a system of \textit{tangent coordinates}. The submanifold is now locally parameterized as $y=(u,g(u))$, where $g:\Real^d \rightarrow \Real^{n-d}$. Since $u=(u_1,\dots,u_d)$ are tangent coordinates, we must have that $\partial g(0)=0$.
\end{enumerate}
Notice that, locally, any function $f$ on $\Man$ may be viewed as $\tilde{f}$ of $(s_1,\dots,s_d)$ and thus we may write $\Delta f(x)=-\sum_{i=1}^d \frac{\partial^2\tilde{f}}{\partial s_i^2}(0,\dots,0)$, where $\Delta$ is the Laplace-Beltrami operator on $\Man$.



\section{Comparison of the Geodesic and the Local Projection}\todo{The following is basically a citation of \cite{Coifman20065}.}
In this section we will compute asymptotic expansions for the changes of variable $u\mapsto (s_1,\dots,s_d)$ and $u\mapsto y$.

In the following, $Q_{x,m}(u)$ denotes a generic homogeneous polynomial of degree $m$ of the variable $u = (u_1,\dots,u_d)$, whose coefficient depends on $x$.

\begin{lemma}
If $y\in\Man$ is in a Euclidean ball of radius $\eps^\frac{1}{2}$ around $x$, then, for $\eps$ sufficiently small, there exists:
\begin{equation}\label{eq:geodOrder3}
s_i=u_i+Q_{x,3}(u)+\mathcal{O}(\eps^2)
\end{equation}
\end{lemma}

\begin{proof}
Let $\gamma$ be the geodesic connection $x$ and $y$ parameterized by arclength. We have $\gamma(0)=x$ and let $s$ be such that $\gamma(s)=y$. If $y$ has normal coordinates $(s_1,\dots,s_d)$, then we have $s\gamma'(0)=(s_1,\dots,s_d)$. A Taylor expansion yields
\begin{equation*}
\gamma(s)=\gamma(0)+s\gamma'(0)+\frac{s^2}{2}\gamma''(0)+\frac{s^3}{6}\gamma^{(3)}(0)+\mathcal{O}(\eps^2).
\end{equation*}
By definition of a geodesic, the covariant derivative of the velocity is zero, which means that $\gamma''(0)$ is orthogonal to the tangent plane at $x$. Now since the parameter $u_i$ is defined by $u_i=\langle\gamma(s)-\gamma(0),e_i\rangle$, we obtain that $u_i=s_i+\frac{s^3}{6}\langle\gamma^{(3)}(0),e_i\rangle+\mathcal{O}(\eps^2)$. Iterating this equation yields the result.
\end{proof}

\begin{lemma}
Again, let $y\in\Man$ be in a Euclidean ball of radius $\eps^\frac{1}{2}$ around $x$; we have
\begin{equation}\label{eq:metricComparison}
||x-y||^2=||u||^2+Q_{x,4}(u)+Q_{x,4}(u)+\mathcal{O}(\eps^3)
\end{equation}
and
\begin{equation}\label{eq:volumeComparison}
det\left(\frac{dy}{du}\right)=1+Q_{x,2}(u)+Q_{x,3}(u)+\mathcal{O}(\eps^2).
\end{equation}
\end{lemma}

\begin{proof}
The submanifold is locally parameterized as $u\mapsto(u,g(u))$, where $g:\Real^d\rightarrow\Real^{n-d}$. Writing $g=(g_{i+1},\dots,g_n)$ and applying Pythagore's theorem, we obtain
\begin{equation*}
||x-y||^2=||u||^2+\sum_{i=d+1}^n g_i(u)^2.
\end{equation*}
Using that, by definition, $g_i(0)=0$ and, as noted before, $\frac{\partial g}{\partial u_i}(0)=0$. As a consequence $g_i(u)=b_{i,x}(u)+c_{i,x}(u)+\mathcal{O}(\eps^2)$, where $b_{i,x}$ is the Hessian quadratic form of $g_i$ at $u=0$ and $c_{i,x}$ is the cubic term. This proves \eqref{eq:metricComparison} with
\begin{equation*}
Q_{x,4}(u)=\sum_{i=d+1}^n b_{i,x}^2(u) \text{ and } Q_{x,5}(u)=2\sum_{i=d+1}^n b_{i,x}(u)c_{i,x}(u).
\end{equation*}

To prove \eqref{eq:volumeComparison}, observe that $\frac{\partial g}{\partial u_i}(0)=0$ implies that $\frac{\partial g}{\partial u_i}(0)=\tilde{b}_{i,x}(u)+\tilde{c}_{i,x}(u)+\mathcal{O}(\eps^\frac{3}{2})$, where $\tilde{b}_{i,x}(u)$ and $\tilde{c}_{i,x}(u)$ are the linear and quadratic terms in the Taylor expansion of $\frac{\partial g}{\partial u_i}(0)$ at $0$. We thus have:

\begin{equation*}\begin{aligned}
\frac{\partial y}{\partial u_i}(u)&=\left(v_i,\frac{\partial g}{\partial u_i}(u)\right), \text{ where } v_i=(0,\dots,0,1,0,\dots,0)\in\Real^d \\
&=(v_i,\tilde{b}_{i,x}(u)+\tilde{c}_{i,x}(u)+\mathcal{O}(\eps^\frac{3}{2}).
\end{aligned}\end{equation*}

The squared volume generated by these $d$ vectors is the determinant of their Gram matrix, i.e.,

\begin{equation*}
\left| det\left(\frac{dy}{du} \right)\right|^2=\sum_{i,j=1}^d E_{ij}(u)+\sum_{i,j=1}^d F_{ij}(u)+\mathcal{O}(\eps^2),
\end{equation*}

where

\begin{equation*}
E_{ij}(u)=\langle \tilde{b}_{i,x}(u),\tilde{b}_{j,x}(u)\rangle \text{ and } F_{ij}(u)=\langle \tilde{b}_{i,x}(u),\tilde{c}_{j,x}(u)\rangle + \langle \tilde{c}_{i,x}(u),\tilde{b}_{j,x}(u)\rangle.
\end{equation*}

Defining

\begin{equation*}
Q_{x,2}(u)=\sum_{i,j=1}^d E_{ij}(u) \text{ and } Q_{x,3}(u)=\sum_{i,j=1}^d F_{ij}(u),
\end{equation*}

we obtain the last result.
\end{proof}



\section{Infinitesimal Operators for a Family of Graph Laplacians}
In this section we present the calculation of the infinitesimal generators for the different diffusion maps characterized by a parameter $\alpha$.

To start with we first show an asymptotic expansion for diffusion operators $G_\eps$.

Let $k_\eps(x,y)$ be an isotropic kernel, i.e.:

\begin{equation*}
k_\eps(x,y)=h\left(\frac{||x-y||^2}{\eps} \right),
\end{equation*}

where $h$ is assumed to have an exponential decay and let $G_\eps$ be the corresponding operator

\begin{equation*}
G_\eps f(x)=\frac{1}{\eps^\frac{d}{2}}\int_\Man k_\eps(x,y)f(y)dy.
\end{equation*}

The idea is that, using the previous lemmata, for small $\eps$ integrating $f$ against the kernel on the manifold is approximately like integrating on the tangent space.

\begin{theorem}\label{thm:diffOpExpansion}
Let $f\in C^3(\Man)$ and let $0 <\gamma<1/2$. Then we have, uniformly for all $x\in\Man$ at distance larger than $\eps^\gamma$ from $\partial\Man$,

\begin{equation*}
G_\eps f(x)=m_0 f(x)+\eps\frac{m_2}{2}\left(\omega(x)f(x)-\Delta f(x) \right)+\mathcal{O}(\eps^2),
\end{equation*}

where

\begin{equation*}
m_0=\int_{\Real^d} h(||u||^2)du \text{ and } m_2=\int_{\Real^d} u_1^2 h(||u||^2)du
\end{equation*}

and $\omega$ is a potential term depending on the embedding of $\Man$.
\end{theorem}

\begin{proof}
Because of the exponential decay of $h$, the domain of integration can be restricted to the intersection of $\Man$ with the ball of radius $\eps^\gamma$ around $x$. In doing so we generate an error of order

\begin{equation*}\begin{aligned}
\Bigg|\frac{1}{\eps^\frac{d}{2}}\bigintsss\limits_{\substack{y\in\Man \\ ||x-y||>\eps^\gamma}} h\left(\frac{||x-y||^2}{\eps} \right)f(y)dy \Bigg|&\leq
||f||_\infty \frac{1}{\eps^\frac{d}{2}}\bigintsss\limits_{\substack{y\in\Man \\ ||x-y||>\eps^\gamma}} \left|h\left(\frac{||x-y||^2}{\eps} \right)\right|dy \\
&\leq ||f||_\infty \int\limits_{\substack{y\in\Man \\ ||y||>\eps^{\gamma-1/2}}} |h(||y||^2)|dy \\
&\leq C||f||_\infty Q(\eps^{1/2-\gamma})e^{-\eps^{\gamma-1/2}},
\end{aligned}\end{equation*}

where we have used the exponential decay of the kernel and where $Q$ is a polynomial. Since $0<\gamma<1/2$, this term is exponentially small and is bounded by $\mathcal{O}(\eps^{\frac{3}{2}})$. Therefore,

\begin{equation*}
G_\eps f(x)=\frac{1}{\eps^\frac{d}{2}}\bigintsss\limits_{\substack{y\in\Man \\ ||x-y||<\eps^\gamma}} h\left(\frac{||x-y||^2}{\eps} \right)f(y)dy+\mathcal{O}(\eps^\frac{3}{2}).
\end{equation*}

Now that things are localized around $x$, we can Taylor-expand the function $(s_1,\dots,s_d)\mapsto f(y(s_1,\dots,s_d))$:

\begin{equation*}
f(y)=f(x)+\sum_{i=1}^d s_i \frac{\partial\tilde{f}}{\partial s_i}(0)+\frac{1}{2}\sum_{i,j=1}^d s_i s_j \frac{\partial^2\tilde{f}}{\partial s_i\partial s_j}(0)+Q_{x,3}(s_1,\dots,s_d)+\mathcal{O}(\eps^2),
\end{equation*}

where $\tilde{f}(s_1,\dots,s_d)=f(y(s_1,\dots,s_d))$. Invoking \eqref{eq:geodOrder3}, we obtain

\begin{equation*}
f(y)=\tilde{f}(0)+\sum_{i=1}^d u_i \frac{\partial \tilde{f}}{\partial s_i}(0)+\frac{1}{2}\sum_{i,j=1}^d u_i u_j \frac{\partial^2\tilde{f}}{\partial s_i\partial s_j}(0)+Q_{x,3}(u)+\mathcal{O}(\eps^2).
\end{equation*}

Likewise, because of \eqref{eq:metricComparison}, the Taylor expansion of the kernel is

\begin{equation*}
h\left(\frac{||x-y||^2}{\eps} \right)=h\left(\frac{||u||^2}{\eps} \right) +\left(\frac{Q_{x,4}(u)+Q_{x,5}(u)}{\eps} \right) h'\left(\frac{||u||^2}{\eps} \right) + \mathcal{O}(\eps^2).
\end{equation*}

Using \eqref{eq:volumeComparison} to change the variable $s\mapsto u$ in the previous integral defining $G_\eps f(x)$ yields:

\begin{equation*}\begin{aligned}
\eps^\frac{d}{2} G_\eps f(x) =& \bigintsss\limits_{||u||<\eps^\gamma} \left[h\left(\frac{||u||^2}{\eps}\right)+\left(\frac{Q_{x,4}(u)+Q_{x,5}(u)}{\eps}\right)h'\left(\frac{||u||^2}{\eps}\right)\right]\\
&\times \left[\tilde{f}(0)+ \sum\limits_{i=1}^d u_i \frac{\partial \tilde{f}}{\partial s_i}(0)+\frac{1}{2}\sum\limits_{i,j=1}^d u_i u_j \frac{\partial^2\tilde{f}}{\partial s_i\partial s_j}(0)+Q_{x,3}(u)\right]\\
&\times \left(1+Q_{x,2}(u)+Q_{x,3}(u)\right)du + \mathcal{O}\left(\eps^{\frac{d}{2}+2} \right).
\end{aligned}\end{equation*}

This identity can be dramatically simplified by identifying odd functions and setting their integral to zero. One is left with

\begin{equation*}\begin{aligned}
\eps^\frac{d}{2} G_\eps f(x)=&\tilde{f}(0)\bigintsss_{\Real^d}h\left(\frac{||u||^2}{\eps} \right)du+\frac{1}{2}\left(\sum_{i=1}^d \frac{\partial^2\tilde{f}}{\partial s_i^2}(0) \right)\bigintsss_{\Real^d}u_1^2 h\left( \frac{||u||^2}{\eps} \right)du\\
+&\tilde{f}(0)\bigintsss_{\Real^d}\left[\frac{Q_{x,4}(u)}{\eps}h'\left(\frac{||u||^2}{\eps}\right)+\tilde{Q}_{x,2}(u) h\left(\frac{||u||^2}{\eps} \right) \right]du+\mathcal{O}\left(\eps^{\frac{d}{2}+2} \right),
\end{aligned}\end{equation*}

where the domain of integration has been extended to $\Real^d$ (exponential decay of $h$). Changing the variable according to $u\mapsto \eps^\frac{1}{2}u$,

\begin{equation*}\begin{aligned}
G_\eps f(x)=&\tilde{f}(0)\int_{\Real^d}h\left(||u||^2 \right)du + \frac{\eps}{2}\left(\sum\limits_{i=1}^d \frac{\partial^2 \tilde{f}}{\partial s_i^2}(0) \right)\int_{\Real^d}u_1^2 h\left(||u||^2\right)du \\
+&\eps\tilde{f}(0)\int_{\Real^d}\left(Q_{x,4}(u)h'\left(||u||^2\right) + Q_{x,2}(u)h\left(||u||^2\right)\right)du + \mathcal{O}(\eps^2),
\end{aligned}\end{equation*}

where we have used the homogeneity of $Q_{x,2}$ and $Q_{x,4}$. Finally, observing that

\begin{equation*}
\tilde{f}(0) = f(x) \text{ and } \sum_{i=1}^d \frac{\partial^2 \tilde{f}}{\partial s_i^2}(0) = -\Delta f(x),
\end{equation*}

we end up with

\begin{equation*}
G_\eps f(x)=m_0 f(x)+\eps\frac{m_2}{2}\left(\omega(x)f(x)-\Delta f(x) \right)+\mathcal{O}(\eps^2),
\end{equation*}

where

\begin{equation*}
\omega(x)=\frac{2}{m_2}\int_{\Real^d}\left(Q_{x,4}(u)h'\left(||u||^2 \right) + Q_{x,2}(u)h\left(||u||^2 \right) \right)du.
\end{equation*}

Finally, the uniformity follows from the compactness and smoothness of $\Man$.
\end{proof}

One can also show the same result holds for all $x\in\Man$ at distance smaller than $\eps^\gamma$ from $\partial\Man$. For this proof we refer to \cite{Coifman20065} as this would go beyond the scope of the discussion.\\

Suppose that the data set X consists of a Riemannian manifold with a density $p(x) = e^{-U(x)}$ and let $k_\eps(x, y)$ be a Gaussian kernel (which clearly satisfies the requirements for \ref{thm:diffOpExpansion}.

\begin{theorem}\label{thm:bFPO}
The infinitesimal generator $\mathcal{H}_b\phi$ of the backward operator $T_{b,\eps}^{(\alpha)}\phi = \int_\Gamma \frac{k_\eps^{(\alpha)}(x,y)}{d_\eps^{(\alpha)}(x)} \phi(y)p(y)dy $ is $\Delta\phi - 2(1-\alpha)\nabla\phi\cdot\nabla U$.
\end{theorem}

\begin{proof}
From \ref{thm:diffOpExpansion} we see that\footnote{w.l.o.g. we assume $m_0=1$ and $m_2=2$} 

\begin{equation*}
p_\eps(x)=p(x)+\eps (\Delta p(x)+ \omega(x)p(x))+\mathcal{O}(\eps^2)
\end{equation*}

and consequently,

\begin{equation*}
p_\eps^{-\alpha}=p^{-\alpha}\left(1-\alpha\eps \left(\frac{\Delta p}{p}+\omega \right) \right)(1+\mathcal{O}(\eps^2)).
\end{equation*}

Let

\begin{equation*}
k_\eps^{(\alpha)}(x,y)=\frac{k_\eps(x,y)}{p_\eps^{\alpha}(x)p_\eps^{\alpha}(y)}
\end{equation*}

Then, the normalization factor $d_\eps^{(\alpha)}$ is given by

\begin{equation*}
d_\eps^{(\alpha)}(x)=\int_\Gamma k_\eps^{(\alpha)}(x,y)p(y)dy = 
p_\eps^{-\alpha}(x)p_\eps^{1-\alpha}(x)\left[1+\eps \left((1-\alpha)\omega - \alpha\frac{\Delta p}{p}+\frac{\Delta p^{1-\alpha}}{p^{1-\alpha}(x)} \right) \right].
\end{equation*}

Therefore, the asymptotic expansion of the backward operator gives

\begin{equation*}
T_b^{(\alpha)}\phi = \int_\Gamma \frac{k_\eps^{(\alpha)}(x,y)}{d_\eps^{(\alpha)}(x)}\phi(y)p(y)dy = \phi(x) + \eps\left(\frac{\Delta(\phi p^{1-\alpha})}{p^{1-\alpha}}-\phi \frac{\Delta p^{1-\alpha}}{p^{1-\alpha}} \right)
\end{equation*}

and its infinitesimal generator is 

\begin{equation*}
\mathcal{H}_b\phi =
lim_{\eps\rightarrow 0}\frac{T_b-I}{\eps}\phi =
\frac{\Delta(\phi p^{1-\alpha})}{p^{1-\alpha}}-\phi \frac{\Delta p^{1-\alpha}}{p^{1-\alpha}}.
\end{equation*}

Plugging in $p=e^{-U}$ into the last equation gives the desired result.
\end{proof}

Similarly, one can show that the infinitesimal generator $\mathcal{H}_f\psi^{(\alpha)}$ of the forward operator $T_{f,\eps}^{(\alpha)}\psi$ is $\Delta\psi - 2\alpha\nabla\psi\cdot\nabla U + (2\alpha - 1)\psi(\nabla U\cdot\nabla U - \Delta U)$.




















\bibliographystyle{te}
\bibliography{references}

\end{document}